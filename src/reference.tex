\part{Reference}

\textbf{A}

绝对路径(absolute path):从根目录开始,包括所有后继子目录的路径。

抽象数据类型(abstract data type):属性(属性和操作)明确地与实现分离的数据类型。

抽象步骤(abstract step):细节仍未明确的算法步骤。

抽象(abstraction):删除了复杂细节的心理模型;复杂系统的一种模型,只包括对观察者来说必需的细节;把数据或动作的逻辑属性与它们的实现细节分离;把对象的属性和它们的实现分离;(在OOD中)从用户角度观察到的对象的关键特征。

访问控制策略(access control policy):一个组织建立的一组规则,规定了接受和拒绝什么类型的网络通信。

读取时间(access time):开始读取一个数据块之前花费的时间,即寻址时间和等待时间之和。

加法器(adder):对二进制值执行加法运算的电路。

可编址性(addressability):内存中每个可编址位置存储的位数。

地址联编(address binding):逻辑地址和物理地址间的映射。

集合运算(aggregate operation):把数据结构作为整体的运算,与针对数据结构的每个组件的运算相对。

算法(algorithm):在有限的时间内用有限的数据解决问题或子问题的明确指令集合。

分配(allocate):在运行时间把内存空间分配给对象使用。

ALU:算术逻辑部件。

模拟数据(analog data):用连续形式表示的信息。

应用软件(application software):帮助我们解决现实世界问题的程序。

实参(arguments):子程序调用中列在括号中的标识符。

算术逻辑部件(arithmetic/logic unit):执行算术运算(加法、减法、乘法和除法)和逻辑运算(两个值的比较)的计算机元件。

数组(array):具有相同类型的元素集合,n(n≥1)位排序;索引n表示元素在某一维中的位置,用n可以直接访问每个元素。

人工智能(artificial intelligence,AI):研究对人类思想建模和应用人类智能的计算机系统的学科。

人工神经网络(artificial neural network):尝试模拟人体神经网络的计算机知识表达法。

汇编器(assembler):把汇编语言程序转化成机器代码的程序。

汇编语言(assembly language):一种低级语言,用助记忆码表示特定计算机的机器语言指令。

断言(assertion):逻辑命题,为true或false。

赋值语句(assignment statement):把一个表达式的值存入一个变量的语句。

异步(asynchronous):不与计算机中的其他操作同时发生;换句话说,即与计算机的动作不同步。

原子数据类型(atomic data type):一种数据类型,只能给该类型的标识符赋一个单一值。

属性(attribute):标记中用于提供有关元素的额外信息的部分。

辅助存储设备(auxiliary storage device):计算机内存以外以编码形式存储数据的设备。

\textbf{B}

带宽(bandwidth):在固定时间内从一个地点传输到另一个地点的最大位数或字节数。

基数(base):基数系统的基础数值,规定了这个系统中使用的数字量和数位位置的值。

基地址(base address):数组中第一个元素的内存地址。

基础情况(base case):在递归解决方案中不必再递归的情况。

基类(base class):可以继承的类。

基址寄存器(base register):存放当前分区的起始地址的寄存器。

大O符号(big-O notation):以函数中随着问题的大小增长得最快的项来表示计算时间(复杂度)的符号。

二进制数字(binary digit):二进制记数系统中的一位数字,可以是0或1。

二进制文件(binary file):包含特定格式的数据的文件,要求给位串一个特定的解释。

二元运算符(binary operation):具有两个运算数的运算符。

二分检索(binary search):在有序列表中检索项目的操作,通过比较操作排除一半检索范围,无论要检索的项目是在列表中的上半部分还是下半部分;这一过程将被反复执行,直到找到了检索的项目,或者确定了该项目不在列表中为止。

位(bit):二进制数字的简称。

块(block):存储在扇区中的信息。

主体(body):循环重复执行的语句;子程序中的可执行语句。

Boolean代数(boolean algebra):表示二值逻辑函数的数学表示法。

Boolean表达式(boolean expression):一个标识符序列,标识符之间由相容的运算符分隔,求得的值是true或false。

Boolean类型(boolean type):由两个值true和false构成的数据类型。

系统引导(booting the system):通过把操作系统载入内存而启动计算机的处理。

界限寄存器(bounds register):存放当前分区的长度的寄存器。

头脑风暴(brainstorming):面向对象设计的初始阶段,将标识出问题中可能的对象类。

分支(branch):并不总是执行的代码段;例如,switch语句和case语句中有多少标记,就有多少分支。

分支控制结构(branching control structure):选择控制结构。

广度优先检索法(breadth-first approach):优先横向检索树的每层,而不是优先向下检索特定路径的检索法。

宽带(broadband):提供的数据传输率大于128Kbps的联网技术。

总线(bus):把机器的主要组成部分连接在一起的一组电线,数据在这组电线中流动。

总线拓扑(bus topology):所有节点共享一根通信线的LAN配置。

字节(byte):8个二进制位。

字节码(bytecode):编译Java源代码使用的标准机器语言。

\textbf{C}

线缆调制解调器(cable modem):使用家庭的有线电视网络进行计算机网络通信的设备。

调用(call):计算机开始执行子程序中的指令的时间点。

化零误差(cancellation error):由于精度限制,当相加或相减的两个数的量级相差太大时发生的精确度损失。

关系约束(cardinality constraint):在ER图中,一次可以存在于两个实体间的关系数量。

区分大小写(case sensitive):大写字母和小写字母被看作是不同的;两个拼写方法相同,但大小写形式不同的标识符被看作两个不同的标识符。

单元格(cell):电子数据表中用于存放数据或公式的元素。

字符集(character set):字符和表示它们的代码的清单。

电路(circuit):相互关联的门的组合,用于实现特定的逻辑函数。

电路等价(circuit equivalence):对应每个输入组合,两个电路都生成完全相同的输出。

循环引用(circular reference):一组公式的计算结果最终错误地互相依赖的情况。

类(一般情况)(class):对一组具有相似属性和行为的对象的描述;(实现阶段)对象的模式。

NP类问题(Class NP problems):用足够多个处理器能在多项式时间内解决的问题。

P类(Class P):由所有多项式时间算法构成的类。

P类问题(Class P problems):用一个处理器能在多项式时间内解决的问题。

客户端(client):声明并操作特定类的对象的软件。

客户/服务器模型(client/server model):客户发出对服务器的请求,服务器做出响应的分布式方法。

代码(code):用程序设计语言编写的计算机指令或数据类型规约。

代码走查(code walk-through):程序的验证过程,将检查程序的每个语句,看它是否如实地实现了对应的算法步骤。

代码覆盖(明箱)测试法(code-coverage(clear-box)testing):通过执行代码中的所有语句测试程序或子程序的测试方法。

编码(coding):把算法翻译成一种程序设计语言;把位组合赋予信息片断的处理。

排序序列(collating sequence):一组或一系列元素(如字符集中的字符(值))的排列次序。

组合电路(combinational circuit):输出仅由输入值决定的电路。

注释(comment):为使用者而写的说明性文字。

编译器(compiler):把用高级语言编写的程序翻译成机器码的程序。

(算法的)复杂度(complexity(of an algorithm)):计算机执行计算相对于计算大小需做的工作的衡量标准。

复合数据类型(composite data type):允许把一组值赋予一个对象的数据类型。

组合(包容)(composition(containment)):一个类的内部数据成员可以定义为另一个类的对象的机制。

压缩率(compression ratio):原始数据大小除压缩后的数据大小。

(电子)计算机(computer(electronic)):能够存储、读取并处理数据的可编程设备。

计算机硬件(computer hardware):计算系统的物理元件。

计算机网络(computer network):为了通信和共享资源而连接在一起的一组计算设备。

计算机程序(computer program):计算机用于解决问题的数据类型规约和执行操作的指令。

计算机程序设计(computer programming):规定计算机用于解决问题的数据类型和操作的处理。

计算机软件(computer software):提供计算机执行的指令的程序。

计算系统(computing system):通过交互来解决问题的计算机硬件、软件和数据。

具体步骤(concrete step):细节完全明确的算法步骤。

条件测试(conditional test):程序中计算boolean表达式,以判断是开始一个新循环,还是跳转到循环后的第一个语句的语句。

常量(constant):程序中的一个项目,它的值是在编译时间固定的,在执行时间不会改变。

常量时间(constant time):大O表达式是个常量的算法。

构造器(constuctor):创建类的新实例的操作;与包容它的类同名的方法,实例化这个类时都会调用它。

容器类(container class):可以加入其它元素的类。

包容(containment):一个类把另一个类的对象作为域的机制。

上下文切换(context switch):当一个进程移出CPU,另一个进程取代了它时发生的寄存器信息交换。

控制抽象(control abstraction):把控制结构的逻辑概观和它的实现分离开。

控制结构(control structure):用于改变正常的顺序控制流的语句;确定程序中的其他指令的执行顺序的指令。

控制器(control unit):控制其他部件的动作,从而执行指令序列的计算机元件。

计数控制循环(count-controlled loop):执行次数预定的循环。

计数器(counter):一个变量,累计它的值可以跟踪处理或事件发生的次数。

中央处理器(CPU):算术逻辑部件和控制器的组合,是计算机用于解释和执行指令的“大脑”。

CPU调度(CPU scheduling):确定主存中的哪个进程可以访问CPU以便执行的动作。

崩溃(crash):计算机的部件故障造成的计算机操作终止;由于错误造成的程序执行终止。

CRC卡(CRC cards):写有类名和它的超类、子类以及责任列表和协作类的索引卡;类、责任和协作类。

光标控制键(cursor control keys):计算机键盘上的一组特殊按键,用户用它们可以在屏幕上上下左右移动光标。

柱面(cylinder):所有磁盘表面的同心磁道的集合。

\textbf{D}

数据(data):以计算机能够使用的形式表示的信息。

数据抽象(data abstraction):把数据的逻辑概观和它的实现分离开。

数据压缩(data compression):减少一段存储数据所需的空间。

数据封装(data encapsulation):把数据的表示法和它在逻辑层的应用分离;强制隐蔽信息的程序设计语言特性。

数据表示法(data representation):用于表示抽象数据类型的抽象值的数据的具体形式。

数据结构(data structure):由用于存储和读取个别数据元素的操作作为结构特征的一组数据元素;抽象数据类型中的复合数据成员的一种实现;抽象数据类型中的复合数据域的一种实现。

数据传输率(也叫带宽)(data transfer rate(also bandwidth)):数据从网络中的一个地点传输到另一个地点的速率。

数据类型(data type):一组值以及能够应用于这种类型的值的基本操作集合的说明。

数据验证(data validation):数据或函数附加的测试,检查数据中的错误。

数据库管理系统(database management system):由物理数据库、数据库引擎和数据库模式构成的软件和数据组合。

数据库(database):结构化的数据集合。

数据覆盖(暗箱)测试法:(data-coverage(black-box)testing):把代码作为一个暗箱,基于所有可能的输入数据测试程序或子程序的测试方法。

释放(deallocate):把分配给对象的存储空间归还到空闲内存池,以便能把它再分配给新对象。

调试(debugging):消除程序中的错误以便使程序能真正完成预计任务的处理。

声明(declaration):把变量、动作或语言中的其他实体与标识符关联起来的语句,使程序员可以通过名字引用这些项目。

深度复制(deep copy):不止把一个对象复制到另一个对象,还要复制所有引用数据的操作。

请求分页(demand paging):页式内存管理法的扩展,只有当页面被引用(请求)时才会载入内存。

降级(缩小)(demotion(narrowing)):根据程序设计语言中数据类型的重要性把高级类型的值转换成低级类型的值;降级可能会损失信息。

深度优先法(depth-first approach):优先沿着树的路径向下检索,而不是优先横向检索每层的检索法。

派生类(derived class):继承其他类的类;在分层体系中是另一个类扩展而来的类。

桌面检查(desk checking):跟踪纸上设计的操作。

开发环境(development environment):具有开发程序需要的所有软件的包。

对话(dialog):用户界面的一种样式,由用户输入数据,并在程序准备处理输入的值前执行一个独立的操作(如点击鼠标按钮)。

数字数据(digital data):用离散形式表示的信息。

数字用户线路(digital subscriber line,DSL):用常规电话线传输数字信号的Internet连接方式。

数字化(digitize):把信息分割成离散的片断。

直接文件访问法(direct file access):通过指定记录编号直接访问文件中的数据的方法。

目录(directory):文件的有名分组。

目录树(directory tree):展示文件系统的嵌套目录组织的结构。

磁盘调度(disk scheduling):决定先满足哪个磁盘I/O请求的操作。

文档(documentation):书面的文本和注释,以便更易于别人理解、使用和修改程序。

文档类型定义(document type definition,DTD):XML文档结构的规约。

域名(domain name):主机名说明组织或分组的部分。

域名服务器(domain name server):把主机名转换成IP地址的计算机。

域名系统(domain name system):管理主机名解析的分布式计算机。

停机(down):说明计算机处于不能使用的状态的描述性术语。

下载(download):在计算机上接受Internet上的信息。

驱动程序(driver):一个简单的伪主程序,用于调用要测试的函数;面向对象程序中的一个主函数。

哑终端(dump terminal):在早期的分时系统中让用户访问主机用的显示器和键盘。

动态分配(dynamic allocation):在运行时间给变量分配内存空间(与在编译时间进行的静态分配相对)。

动态联编(dynamic binding):在运行时间决定调用多态方法的哪种形式。

动态内存管理(dynamic memory management):在执行应用程序的过程中根据需要分配或释放存储空间。

动态分区法(dynamic-partition technique):根据容纳程序的需要对内存分区的内存管理方法。

\textbf{E}

回应打印(echo printing):打印程序的输入数据以验证它们是否正确。

编辑器(editor):用于创建和修改源程序或数据的交互式程序。

有效权(effective weight):人工神经元中输入值和相应的权的乘积之和。

封装(encapsulation):实施信息隐蔽的语言特性;把数据和行为集中在一起,使数据和行为的逻辑属性与它们的实现细节分离。

实体关系(ER)建模法(entity-relationship(ER)modeling):设计关系数据库的常用方法。

ER图(ER diagram):ER模型的图形化表示法。

以太网(ethernet):基于总线拓扑的局域网业界标准。

求值(evaluate):对指定的值执行一组特定的运算以计算一个新值。

事件(event):与程序执行异步的动作,如点击鼠标。

事件计数器(event counter):每当一个特定事件发生一次就累加1的变量。

事件处理器(event handler):事件侦听器的一个方法,当侦听器接收到响应事件时将调用该方法。

事件处理(event handling):对在程序执行过程中随时可能发生的事件的响应处理。

事件侦听器(event listener):等待一个或多个事件发生的对象。

事件控制循环(event-controlled loop):当循环主体中发生了某事表示循环应该退出时就终止的循环。

异常(exception):程序执行过程中检测到的不正常状况;抛出异常将终止方法的正常执行。

异常处理器(exception handler):在Java或C++程序中发生了异常时执行的一段程序。

执行(executing):计算机根据给定的程序执行的动作。

执行跟踪(execution trace):用真正的值遍历一次程序,记录每个变量的状态。

专家系统(expert system):基于人类专家的知识构造的软件系统。

表达式(expression):标识符、文字和运算符的排列,通过求值可以计算指定类型的值。

表达式语句(expression statement):表达式加分号构成的语句。

可扩展标记语言(extensible markup language,XML):允许用户描述文档内容的语言。

可扩展样式表语言(extensible stylesheet language,XSL):定义XML文档到其他输出格式之间转换的语言。

外部文件(external file):用于与人和程序通信的文件,存储在程序之外。

外部指针(external pointer):引用链表中的第一个节点的有名指针变量。

外部表示法(external respresentation):数值可打印的(字符)形式。

\textbf{F}

读取-执行周期(fetch-execute cycle):中央处理器为每条机器语言指令执行的一系列步骤。

域(fields):类中的特定项,可以是数据或子程序。

文件(file):数据的有名集合,用于组织辅助存储设备。

文件扩展名(file extension):文件名中说明文件类型的那部分。

文件服务器(file server):专用于为网络用户存储和管理文件的计算机。

文件系统(file system):操作系统为它管理的文件提供的逻辑视图。

文件类型(file type):文件(如程序或文档)中存放的关于类型的信息。

过滤(filtering):面向对象设计的一个阶段,头脑风暴阶段提出的对象类将在这个阶段被细化,此外还要加入遗漏的对象类。

有限状态机(finite state machine):简单计算机的模型,由一组状态、规定状态何时转变的规则和状态改变时执行的一组动作构成。

防火墙(firewall):一台网关机器,它的软件通过过滤网络通信来保护网络。

引发事件(firing an event):生成事件的事件源。

固定分区法(fixed-partition technique):把内存分成特定数目的分区以载入程序的内存管理方法。

标志(flag):在程序的某处设置,在另一处测试,以控制程序的逻辑流的boolean变量。

浮点表示法(floating point):标明了符号、尾数和指数的实数表示法。

控制流(flow of control):程序中的语句的执行顺序。

格式化(formatting):计划的语句或声明以及程序中的空行的位置;程序输出的安排,以便使输出中的空格和对齐方式整齐美观。

帧(frame):大小固定的一部分主存,用于存放进程页。

全加器(full adder):计算两个数位的和,并考虑进位的电路。

功能内聚性(functional cohesion):模块的一种属性,其中所有固定步骤都只解决一个问题,重要的子程序都被写作抽象步骤。

功能分解(functional decomposition):软件开发的一种方法,把问题分解成更易于解决的子问题,这些子问题的解决方案将构成整体问题的解决方案;自顶向下设计法相似。

功能等价性(functional equivalence):模块的一种属性,执行的操作恰好是它定义的抽象步骤,或者一个模块执行的操作与另一个模块执行的完全一样。

功能模块(functional modules):自顶向下设计中的结构化任务和子任务,单独解决这些任务可以创建一个有效的程序。

功能型问题说明(functional problem description):明确地陈述了程序要做什么的说明。

\textbf{G}

门(gate):对电信号执行基础运算的设备,接受一个或多个输入信号,生成一个输出信号。

网关(gateway):处理它的LAN和其他网络之间通信的节点。

一般(递归)情况(general(recursive)case):递归解决方案中的一种情况,由问题的更小版本来表示。

\textbf{H}

半加器(half adder):计算数位的和,并生成正确进位的电路。

停机问题(halting problem):确定对于指定的输入一个程序最终是否能停止的问题,是不可解决的。

硬件(hardware):计算机的物理硬件。

试探法(heuristics):各种各样的问题求解策略。

分层结构(hierarchy):抽象的结构,其中子孙对象将继承它的祖先的特征。

高级程序设计语言(high-level programming language):一条语句将被翻译成一条或多条机器语言指令的程序设计语言。

同构(homogeneous):说明所有元素的数据类型相同的结构(如数组)的术语。

主机号(host number):IP地址中指定特定网络主机的那部分。

主机名(hostname):由点号分隔的单词组构成的名字,唯一标识了Internet上的机器;每个主机名对应一个IP地址。

Huffman编码(huffman encoding):用变长的二进制串表示字符,使常用的字符具有较短的编码。

超文本标示语言(hypertext markup language,HTML):用于创建Web页的语言。

\textbf{I}

标识符(identifier):包、类、方法或域的名字,用于引用它们。

实现阶段(implementation phase):计算机程序设计的第二个阶段;把算法翻译(编码)成程序设计语言;在计算机上运行生成的程序以进行测试,检查它的正确性,进行必要的修改;使用程序。

实现(implementing):编码并测试程序。

实现测试计划(implementing a test plan):用测试计划中列出的测试用例运行程序。

索引(index):用于选择数组中的元素的值。

推理机(inference engine):处理规则以得出结论的软件。

无限循环(infinite loop):终止条件永远也不能满足,从而在没有程序以外的干涉的情况下永远不能退出的循环。

无限递归(infinite recursion):由于没有达到基础情况,所以子程序不断调用自身的情况。

信息(information):可以传递的消息。

信息隐蔽(information hiding):隐蔽模块的细节以控制对这些细节的访问的做法。

信息系统(information system):帮助我们组织和分析数据的软件。

继承(inheritance):一个类获得了另一个类的属性(即数据域和方法)的机制;类分层结构的一种设计技术,其中每个子孙类都继承了它的祖先类的属性(数据和操作);是我们能利用现有类的定义来定义新类的机制。

输入(input):把外部数据集合中的值存入程序中的变量的处理;数据来自输入设备(键盘)或辅助存储设备(硬盘或磁带)。

输入提示(input prompts):由交互式程序打印出的消息,解释需要输入什么数据。

输入部件(input unit):接收要存储在内存中的数据的设备。

输入/输出(I/O)设备(input/output(I/O)devices):接收要处理的数据(输入)以及呈现处理结果(输出)的计算机部件。

审查(inspection):一种验证方法,其中一个小组成员要逐行读出程序或设计,其他成员负责指出其中的错误。

实例化(instantiate):创建类的对象。

指令寄存器(instruction register,IR):存放当前执行的指令的寄存器。

整数(integer):自然数、自然数的负数和0。

集成电路(又称芯片)(integrated circuit):嵌入了多个门的硅片。

交互式系统(interactive system):使用户能够与计算机直接通信的系统。

internet:遍布地球的广域网。

internet骨干网(internet backbone):承载internet通信的一组高速网络。

网际协议(internet protocol,IP):处理包通过互相连接的网络传递到最终目的的路由选择的网络协议。

internet服务提供者(internet service provider,ISP):提供internet访问的公司。

互通性(interoperability):多台机器上的来自多个销售商的软件和硬件互相通信的能力。

解释器(interpreter):输入用高级语言编写的程序,指导计算机执行每个语句指定的动作的程序。

调用(invoke):调用一个子程序,在控制流返回调用后的语句之前执行子程序。

IP地址(IP address):由点号分隔的四个数值构成的地址,唯一表示了internet上的机器。

迭代(iteration):循环主体的一遍经历或一次重复。

迭代计数器(iteration counter):每次循环迭代就累加1的计数器变量。

\textbf{J}

Java小程序(Java applet):为嵌入HTML文档而设计的程序,能够通过Web传输,在浏览器中执行。

JSP小脚本(JSP scriplet):嵌在HTML文档中用于给Web页提供动态内容的代码片段。

键(key):在表的所有记录中唯一标识一个数据库记录的一个或多个域。

关键字编码(keyword encoding):用单个字符代替常用的单词。

基于知识的系统(knowledge-based system):使用特定信息集合的软件。

\textbf{L}

等待时间(latency):把指定的扇区定位到读写头之下所花费的时间。

长度(length):列表中的项目个数;长度可以随着时间而改变。

词法二义性(lexical ambiguity):由于单词具有多种含义而造成的二义性。

生存期(lifetime):对变量、常量或对象来说,是它们占有内存存储空间这段时间中部分程序的执行时间。

线性关系(linear relationship):除了第一个元素外,每个元素都有唯一的前驱元素,除了最后一个元素外,每个元素都有唯一的后继元素。

线性时间(linear time):算法的大O符号是一个常量乘以n的表达式,其中n是数据集中的数值个数。

链接(link):两个Web页之间的连接。

链表(linked list):元素的顺序由每个节点的链接域决定,而不是由元素在内存中的排列顺序决定的列表。

直接量值(literal value):程序中的常量值。

装入程序(loader):软件用于读取机器语言程序并把它载入内存的部分。

局域网(local-area-network,LAN):连接较小地理范围内的少量计算机的网络。

Loebner奖(Loebner prize):第一个正式的图灵测试,每年举行一次。

对数量级(logarithmic order):算法的大O符号是用n的对数表达的,其中n是数据集中的数值个数。

退出系统(logging off):通常用简单的命令通知计算机没有要执行的命令了。

进入系统(logging on):必要的准备步骤,让计算机识别用户的身份,以便它能够接收用户的命令。

逻辑框图(logic diagram):电路的图形化表示,每种类型的门有自己专用的符号。

逻辑地址(logical address):对一个存储值的引用,是相对于引用它的程序的。

逻辑顺序(logical order):程序员希望的程序语句的执行顺序,可能不同于它们出现的物理顺序。

循环(loop):组织语句的一种方法,在满足某种条件的情况下将反复执行其中的语句。

循环入口(loop entry):控制流首次进入循环内部的语句的点。

循环出口(loop exit):循环主体的反复执行结束,控制流传递到循环后的第一个语句的点。

循环测试(loop test):计算循环表达式,决定是开始新一轮迭代,还是跳转到紧接着循环的语句的点。

无损压缩(lossless compression):不会丢失信息的数据压缩技术。

有损压缩(lossy compression):会丢失信息的数据压缩技术。

\textbf{M}

机器语言(machine language):由计算机直接使用的二进制编码指令构成的语言。

主机(mainframe):大型的多用户计算机,通常采用早期的分时系统。

维护(maintenance):在程序完成之后修改程序,使它满足修改过的要求或消除其中的错误。

维护阶段(maintenance phase):进行维护的时期。

尾数(mantissa):在实数的浮点数表示法中,表示数字自身,而不是它的指数数位。

标记语言(markup language):使用标记来注释文档中的信息的语言。

内存管理(memory management):了解主存中载有多少个程序以及它们的位置的动作。

内存单元(memory unit):计算机的内部数据存储空间。

元语言(metalanguage):用于定义其他语言的语言。

方法(method):定义了类的一种行为的特定算法。

城域网(metropolitan-area network,MAN):为大城市开发的网络基础设施。

MIME类型(MIME type):定义emai附件或网站文件的格式的标准。

模型(model):真实系统的抽象;系统中的对象和管理对象相互作用的规则的表示法。

模块(module):解决问题或子问题的封闭步骤集合。

主板(motherboard):个人计算机的主电路板。

多媒体(multimedia):几种不同的媒体类型。

多路复用器(multiplexer):使用一些输入控制信号决定用哪条输入数据线发送输出信号的电路。

多道程序设计(multiprogramming):同时在主存中驻留多个程序,由它们竞争CPU的技术。

\textbf{N}

有名常数(named constant):由标识符引用的一个内存单元,存放的数据值不能改变。

自然语言(natural language):人们用于交流的语言。

自然语言理解(natural language comprehension):用计算机对人类传达的信息做出合理的解释。

自然数(natural number):0或通过在0上重复加1得到的任何数。

负数(negative number):小于0的数,是在相应的正数前面加上负号。

嵌套控制结构(nested control structure):由一个控制语句(选择、迭代或子程序)嵌入另一个控制语句构成的程序结构。

网络地址(network address):IP地址中指定特定网络的那部分。

节点(或主机):(node(or host)):网络中任何可寻址的设备。

节点(nodes):动态结构的构建块,每个块由元素(数据)或指向下一个节点的指针(链接)构成。

非抢先调度(nonpreemptive scheduling):当当前执行的进程自愿放弃了CPU时发生的CPU调度。

NP完全问题(NP-complete problems):NP类问题的子集,如果发现了其中任何一个问题的单处理器多项式时间的解决方案,那么其他所有问题都存在这样的解决方案。

数字(number):抽象数学系统的一个单位,服从算术法则。

\textbf{O}

对象(object):在问题背景中相关的事物或实体。

对象(问题求解阶段)(object(problem-solving phase)):与问题背景相关的事物或实体。


对象类或类(问题求解阶段)(object class or Class(problem-solving phase)):属性和行为相似的一组对象的说明。

目标码(object code):源代码的机器语言版本。

目标程序(object program):源程序的机器语言版本。

基于对象的程序设计语言(object-based programming language):支持抽象和封装机制,但不支持继承机制的程序设计语言。

面向对象的设计(object-oritened design):软件开发的一种技术,其中解决方案由对象表示。所谓对象,是由数据和通过传递消息来操作数据的封闭实体。

一维数组(one-dimensional array):同种类型的元素的结构化集合,具有指定的名字;通过说明元素在集合中的位置的索引可以直接访问每个元素。

开放式系统(open system):以网络体系结构的通用模型为基础并且伴有一组协议的系统。

开放式系统互联参考模型(open system interconnection reference model):为了便于建立网络标准而对网络交互进行的7层逻辑划分。

操作系统(operating system):管理计算机资源并为系统交互提供界面的系统软件。

越界数组索引(out-of-bounds array index):小于数组中的第一个元素的位置或大于最后一个元素的位置的索引值。

输出部件(output unit):一种设备,用于把存储在内存中的数据打印或显示出来,或者把存储在内存或其他设备中的信息制成一个永久副本。

溢出(overflow):当计算机的结果太大以至于给定的计算机不能表示时发生的情况。

\textbf{P}


包(packet):在网络上传输的数据单位。

包交换(packet switching):把包单独发送到目的地然后再组装起来的网络通信技术。

页(page):大小固定的一部分进程,存储在内存帧中。

页映射表(page map table,PMT):操作系统用于记录页和帧之间的关系的表。

页面交换(page swap):把一个页面从辅助存储设备载入内存,通常会使另一个页面从内存中删除。

页式内存管理法(paged memory technique):把进程划分为大小固定的页,载入内存时存储在帧中的内存管理方法。

形参(parameter):列在子程序名后的括号中的标识符。

形参列表(parameter list):程序中两部分之间的通信机制。

形参传递(parameter passing):在子程序调用中值参和形参之间直接的数据传输。

地址传递(pass by address):一种形参传递机制,传递给正式参数的是真正参数的内存地址;也叫做引用传递。

值传递(pass by value):一种形参传递机制,传递给正式参数的是真正参数的值的副本。

密码(password):赋予用户的唯一字符序列(只有用户才知道),用户在登录过程中可以用它向计算机证明自己的身份;密码系统可以保护计算机中存储的信息,使它们不被篡改或破坏。

路径(path):文件或子目录在文件系统中的位置的文本名称。

周边设备(peripheral device):隶属于计算机的输入设备、输出设备或辅助存储设备。

个人计算机(personal computer,PC):主要为个人使用而设计的小型计算机系统(通常放于桌面上)。

电话调制解调器(phone modem):把计算机信号转换成模拟音频信号,然后再把模拟音频信号准换回计算机信号的设备。

音素(phonemes):任何指定的语言中的基础声音单元的集合。

物理地址(physical address):主存储设备中的真实地址。

Ping:用于测试一台特定的网络计算机是否是活动的以及是否可到达的程序。

流水线操作(pipelining processing):一前一后地安排多个处理器,使每个处理器负责整个运算的一部分。

像素(pixels):用于表示图像的独立点,代表图像的元素。

多态性(polymorphism):一种语言的继承体系结构中具有两个同名方法,而这种语言能够根据对象应用合适的方法的能力。

多项式时间算法(polynomial-time algorithm):复杂度能用问题大小的多项式表示的算法。

端口(port):特定高级协议对应的数字标号。

位置记数法(positional nonation):一种表达数字的系统,数位按顺序排列,每个数位有一个位值,数字的值是每个数位和位值得乘积之和。

后缀运算符(postfix operator):位于运算数之后的运算符。

精度(precision):最多可以表示的有效位数。

前置条件(preconditions):在执行模块前必须为true的断言。

抢先调度(preemptive scheduling):当操作系统决定照顾另一个进程,抢占当前执行进程的CPU资源时发生的CPU调度。

前缀运算符(prefix operator):位于运算数之前的运算符。

问题求解(problem solving):找到令人感到迷惑混乱的难题的解决方案的行动。

问题求解阶段(problem solving phase):计算机程序设计的第一组步骤:分析问题;开发算法;测试算法的正确性。

过程抽象(procedural abstraction):把动作的逻辑概观和它的实现分离开。

进程(process):程序执行过程中的动态表示法。

进程控制块(process control block,PCB):操作系统管理进程信息使用的数据结构。

进程管理(process management):了解活动进程的信息的动作。

进程状态(process states):在操作系统管理下,进程历经的概念性状态。

程序(program):用于执行特定任务的指令序列。

程序计数器(program counter,PC):存放下一条要执行的指令的地址的寄存器。

程序设计(programming):计划、调度或执行一个任务或事件。

程序设计语言(programming language):用于构造程序——表示成计算机能够理解的指令序列——的规则、符号和专用字集合。

专有系统(proprietary system):使用特定销售商的私有技术的系统。

协议(protocol):定义如何在网络上格式化和处理数据的一组规则。

协议栈(protocol stack):彼此依托的协议分层。

伪代码(pseudocode):英语语句和控制结构的混合物,可以很容易地翻译成编程语言。

脉冲编码调制(pulse-code modulation):电信号在两个极端之间跳跃的变化。

\textbf{Q}

查询(query):提交给数据库的信息请求。

\textbf{R}

小数点(radix point):在记数系统中,把一个实数分割成整数部分和小数部分的点。

范围(range):用端点指定的一组连续单元格。

值的区间(range of values):值所属的区间,用最大可用值和用最小可用值指定。

光栅图形格式(raster-graphics format):逐个像素存储图像信息的格式。

有理数(rational number):整数或两个整数的商(不包含被0除的情况)。

实数(real number):由整数部分和小数部分构成的数字,没有虚数部分。

实时系统(real-time system):应用程序的特性决定了响应时间至关重要的系统。

重新计时(reclock):在信号降级太多之前将它重置为原始状态的行为。

记录(或对象、实体)(record(or object,entity)):构成一个数据库实体的相关的域的集合。

递归(recursion):子程序调用自身的能力。

递归调用(recursive call):一种子程序调用,被调用的子程序与调用它的子程序相同。

递归情况(recursive case):与一般情况有关。

递归定义(recursive definition):用自身的更小版本定义自己的定义方法。

引用参数(reference parameter):由调用部件传入实参的地址(写在留言板上)的形参。

指代二义性(referential ambiguity):由于代词可以指代多个对象而造成的二义性。

细化(refinement):在自顶向下设计中,扩展模块定义以构成一个新模块,作为问题的计算机解决方案中的一个主要步骤。

寄存器(register):CPU上的一小块存储区域,用于存储中间值或特殊数据。

关系模型(relational model):用表组织数据和数据之间的关系的数据库模型。

关系操作符(relational operator):声称在两个值之间存在一个关系的操作符。

相对路径(relative path):从当前工作目录开始的路径。

中继器(repeater):在较长的通信线路上加强和传播信号的网络设备。

表示(舍入)误差(representational(round-off)error):由于算术运算结果的精度大于机器的精度造成的算术误差。

保留字(reserved word):一种语言中具有特殊意义的字,不能用它作为标识符。

分辨率(resolution):用于表示图像的像素个数。

响应时间(response time):收到信号和生成响应之间的延迟时间。

责任算法(responsibility algorithm):面向对象设计中的类方法的算法;设计过程中开发算法的阶段。

return:计算机从执行的子程序返回的点。

重用性(reuse):在多个程序中使用一个类,而无需修改类或程序。

右对齐(right-justified):把字符放在距离右边界固定字符数的位置上。

环形拓扑(ring topology):所有节点连接成了封闭环的LAN配置。

强壮的(robust):描述程序能从错误的输入恢复,并继续运行的术语。

根目录(root directory):包含其它所有目录的最高层目录。

路由器(router):指导包在网络上向最终目的地传输的路线的网络设备。

基于规则的系统(rule-based system):基于一套if-then规则的软件系统。

行程长度编码(run-length encoding):把一系列重复字符替换为它们重复出现的次数。

\textbf{S}

场景(scenarios):面向对象设计的一个阶段,将给类分配责任。

模式(schema):数据库中的数据的逻辑结构的规约。

科学记数法(scientific notation):另一种浮点表示法。

访问域(作用域)(scope of access(scope)):能够合法引用(使用)一个标识符的程序代码段。

作用域规则(scope rules):根据声明标识符的位置和它的特定转义符确定在程序的哪些地方才可以引用一个标识符的规则。

检索树(search tree):表示对抗性情况(如博弈)中的所有选择的结构。

二级存储设备(secondary storage device):辅助存储设备。

扇区(sector):磁道的一个区。

寻道时间(seek time):读写头定位到指定的磁道所花费的时间。

选择控制结构(selection control structure):程序结构的一种形式,允许计算机根据给定的背景在几种可能的动作选择一种来执行;也叫做分支控制结构。

自编文件代码(self-documenting code):包含标识符和用于使标识符更易于理解的注释的程序。

语义网络(semantic network):表示对象之间关系的知识表达法。

语义(semantics):赋予一种程序设计语言中的指令含义的一套规则。

半导体(semiconductor):既不是良好的导体,也不是绝缘体的材料,如硅。

标记(sentinel):某些事件控制循环中使用的特殊数值,用于标识循环应该退出了。

序列(sequence):其中的语句一个接一个被执行的结构。

时序电路(sequential circuit):输出是输入值和电路当前状态的函数的电路。

顺序文件访问法(sequential file access):以线性方式访问文件中的数据的方法。

浅复制(shallow copy):把一个类的对象复制到另一个对象,但并不复制所引用的数据的操作。

共享内存(shared memory):多个处理器共享一个全局内存。

简化(条件)估算(short-circuit(conditional)evaluation):按照从左到右的顺序计算逻辑表达式,一旦确定了最后的boolean值,计算就停止。

有效位(significant digits):从左边的第一个非零数位开始,到右边的最后一个非零数位(或纯粹的零)结束的数字。

符号数值表示法(sign-magnitude representation):符号表示数所属的分类(正数或负数),值表示数的量值的数字表示法。

模拟(simulation):设计复杂系统的模型并为观察结果而对该模型进行实验。

单块内容管理(single contiguous memory management):把应用程序载入一段连续的内存区域的内存管理方法。

(数组的)大小:(size(of an array)):为数组保留的物理空间。

软件(software):计算机程序;计算机上可用的所有程序的集合。

软件工程(software engineering):传统的工程化方法和技术在软件开发方面的应用。

软件生命周期(software life cycle):大型软件项目生命的各个阶段,包括需求分析、规约、设计、实现、测试和维护。

软件盗版(software piracy):个人使用或其他多人使用的未授权软件副本。

软件需求(software requirements):说明计算机系统或软件产品提供的功能的语句。

软件规约(software specification):软件产品的功能、输入、处理、输出和特性的详细说明;提供了设计和实现软件所必需的信息。

排序关键字(sort key):用于排序的域。

有序列表(sorted list):列表中的项与前驱项和后继项之间的关系由该项的关键字决定的列表;列表项的关键字之间有语义关系。

排序(sorting):把清单中条目按照数字顺序或字母顺序排序。

源程序(source program):用高级程序设计语言编写的程序。

空间压缩(spatial compression):基于静态图像的压缩方法的电影压缩方法。

电子制表软件(spreadsheet):允许用户用单元格组织和分析数据的程序。

电子数据表函数(spreadsheet function):电子制表软件提供的可用于公式的计算函数。

稳定排序(stable sort):保留重复项的顺序的排序算法。

标准化的(standardized):统一的;大多数高级语言都是标准化的,存在官方说明。

星形拓扑(star topology):由中心节点控制所有消息传输的LAN配置。

(一般的)字符串(string(general sense)):字符序列,如双引号中的单词、名字或句子。

强等价性(strong equivalence):两个系统基于结果和实现这种结果的处理方法的等价性。

强类型化(strong typing):每个变量都有一个类型,只有这种类型的值才能存储到该变量中。

结构化数据类型(structed data type):有组织的元素集合;这种组织法决定了访问单个元素的方法。

结构化查询语言(structed query language,SQL):用于管理和查询数据的综合性关系数据库语言。

风格(style):程序员把算法翻译成程序的个人习惯。

超级计算机(supercomputer):功能最强大的一类计算机。

同步处理(synchronous processing):用多个处理器把同一个程序同时应用到多个数据集。

句法二义性(syntactic ambiguity):由于句子的构造方式有多种而造成的二义性。

语法(syntax):规定有效指令的结构的正式规则。

系统软件(system software):管理计算机系统并与硬件进行交互的程序。

\textbf{T}

表(table):数据库记录的集合。

标记(tag):标记语言中用于说明如何显示信息的语法元素。

尾递归法(tail recursion):递归调用返回后不再执行任何语句的递归算法。

TCP/IP:一组支持低层网络通信的协议和程序。

团队编程方法(team programming):对于一个程序员要花很长时间才能完成的程序,用两个或更多个程序员来设计的方法。

时间压缩(temporal compression):根据连续帧之间的差别压缩电影的方法。

十进制补码(ten’s complement):一种负数表示法,负数I用10的k次幂减I表示。

终止条件(termination condition):使循环退出的条件。

测试计划(test plan):说明如何测试程序的文档。

测试(testing):把程序的输出与手动计算的结果进行比较以检查程序输出的方法;用设计好的数据集运行程序以发现程序中的错误。

测试计划实现(test-plan implementation):用测试计划中规定的测试用例验证程序是否输出了预期的结果。

文本文件(text file):包含字符的文件。

系统颠簸(thrashing):频繁的页面交换造成的低效处理。

抛出(throw):发信号说明发生了异常的动作;抛出一个异常,不正常地终止子程序的执行。

时间片(time slice):在CPU循环调度算法中分配给每个进程的时间量。

分时系统(timesharing):CPU时间由多个同时进行交互的用户共享的系统。

自顶向下设计法(top-down design):把问题分解成多个更易于解决的子问题,再把每个子问题的解决方案组合起来构成整体问题的解决方案的程序开发方法。

顶级域名(top-level domain,TLD):域名中的最后一部分,声明了组织的类型或所属国家。

跟踪路由程序(traceroute):用于展示包在到达目的节点的过程中经过的路线的程序。

磁道(track):磁盘表面的同心圆。

训练(training):调整神经网路中的权和阀值以实现想要的结果的过程。

传送速率(transfer rate):数据从磁盘传输到内存的速率。

晶体管(transistor):作为导线或电阻器的设备,由输入信号的电压电平决定它的作用。

传输控制协议(transmission control protocol,TCP):把消息分割成包,在目的地把包重新组装成消息,并负责处理错误的网络协议。

遍历列表(traverse a list):从头到尾访问列表的元素。

真值表(truth table):列出了所有可能的输入值和相关的输出值的表。

图灵测试(turing test):一种行为方法,用于判断一个计算机系统是否是智能的。

周转周期(turnaround time):从进程进入准备状态到它完成之间的时间间隔,是评估CPU调度算法的标准之一。

二维数组(two-dimensional array):类型相同,具有二维结构的元素集合;每个元素由一个表示元素在每维中的位置的索引对访问。

选型(类型转换)(type casting(type conversion)):把一个值明确地从一种数据类型转换为另一种类型。

强制类型转换(type coercion):把一种类型的值自动转换为另一种类型。

\textbf{U}

一元运算符(unary operator):只有一个运算数的运算符。

下溢(underflow):当计算的结果太小以至于给定的计算机不能表示时发生的情况。

统一资源地址(uniform resource locator,URL):说明Web地址的标准方式。

非结构化数据类型(unstructured data type):由无组织的元素构成的结合。

上载(upload):从计算机给internet上的目标机器发送数据。

用户名(user name):计算机用于识别用户的名字,在登录计算机时必须输入。

用户数据报协议(user datagram protocol,UDP):牺牲一定可靠性以实现较高传输速率的网络协议,是TCP的替代者。

\textbf{V}

值参(value parameter):由调用部件传入实参的副本(写在留言板上)的形参。

返回值的函数(value-returning function):在表达式中调用,给调用者返回一个值的函数(子程序)。

变量(variable):由存放数值的标识符引用的内存单元。

矢量图形(vector graphics):用线段或几何形状表示图像的方法。

视频编译码器(video codec):用于缩减电影大小的方法。

虚拟计算机(virtual computer(machine)):用于说明真实计算机的重要性的假想机。

虚拟机(virtual machine):分时系统创建的每个用户拥有一台专用机器的假象;由于整个过程不必同时处于内存而造成的对程序大小没有限制的假象。

病毒(virus):能够自我复制的计算机程序,其目标通常是在未经授权的情况下传播到其他计算机,可能具有破坏的意图。

语音识别(voice recognition):用计算机来识别人类所讲的话。

语音合成(voice synthesis):用计算机制造出人类的声音。

声波纹(voiceprint):表示人声随着时间推移的频率变化的图。

\textbf{W}

走查(walk-through):由一个小组手动地模拟程序或设计的验证方法。

弱等价性(weak equivalence):两个系统基于结果的等价性。

Web浏览器(Web browser):获取并显示Web页的软件工具。

Web页(Web page):包含或引用各种类型的数据的文档。

Web服务器(Web server):专用于响应网页请求的计算机。

Web站点(Web site):一组相关的网页,通常由同一个人或公司设计和控制。

模拟假设分析(what-if analysis):修改电子数据表中表示假设的值,以观察假设的变化对相关数据有什么影响。

广域网(Wide-area network,WAN):连接两个或多个局域网的网络。

无线连接(wireless):没有物理电线的网络连接。

字(word):一个或多个字节,字中的位数称为计算机的字长。

工作(work):计算机执行计算所做的处理量。

工作目录(working directory):当前活动的子目录。

工作站(workstation):主要为单人使用设计的小型机或功能强大的微型机。

万维网(或Web)(world wide web(or Web)):信息和用于访问信息的网络软件的基础设施。


